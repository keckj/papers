

\documentclass{beamer}

\usepackage[T1]{fontenc}
\usepackage[utf8]{inputenc}
\usepackage{lmodern}
\usepackage[french,english]{babel}

\usepackage{amsthm}
\usepackage{float}
\usepackage{lmodern}%pour un meilleur rendu des polices
\usepackage{verbatim}%du texte non interprt
\usepackage[cmex10]{amsmath}
\usepackage{amssymb}%maths
\usepackage{xspace}
\usepackage[dvipsnames,svgnames,table]{xcolor}
\usepackage{listings}
\usepackage{fancyhdr}
\usepackage{etoolbox}
\usepackage{titlesec}
\usepackage{titletoc}
\usepackage{lastpage}
\usepackage[bookmarks=true,bookmarksnumbered=true]{hyperref}
\usepackage{ctable} % for \specialrule command
\usepackage{cite}
\usepackage{algorithm2e}
\usepackage{alltt}
\usepackage{array}
\usepackage{mdwmath}
\usepackage{mdwtab}
\usepackage{eqparbox}
\usepackage[caption=false,font=normalsize,labelfont=sf,textfont=sf]{subfig}
\usepackage{dblfloatfix}
\usepackage{url}
\usepackage{tipa}
\usepackage{stmaryrd}

\usetheme{Warsaw}

\setbeamercovered{transparent}


\newcommand{\ccite}[1]{\textbf{\cite{#1}}}
\newcommand{\pd}[2]{\dfrac{\partial #1}{\partial #2}}
\newcommand{\od}[2]{\dfrac{\mathscr{D}_a #1}{\mathscr{D} #2}}
\newcommand{\tensor}[1]{\mathbf{#1}}
\renewcommand{\vector}[1]{\overrightarrow{#1}}

\newcommand{\Tau}{\tensor{\mathlarger{\uptau}}}
\renewcommand{\v}{\vector{u}}
\newcommand{\W}{\tensor{W\left( \v \right)}}
\newcommand{\D}{\tensor{D\left( \v \right)}}
\newcommand{\grad}{\vector{\nabla}}
\newcommand{\gradv}{\tensor{\nabla\v}}
\renewcommand{\div}[1]{div \left( #1 \right)}
\newcommand{\divv}[1]{\vector{div} \left( #1 \right)}
\newcommand{\UU}{\mathcal{U}}
\newcommand{\VV}{\mathcal{U}}
\newcommand{\M}{\mathcal{M}}
\newcommand{\A}[2]{\mathcal{A}_{#1}\left( #2 \right)}
\newcommand{\Uu}{\UU^{n}}
\newcommand{\Uv}{\UU^{n+\theta}}
\newcommand{\Uw}{\UU^{n+1-\theta}}
\newcommand{\Ux}{\UU^{n+1}}
\newcommand{\Uk}{\UU^{k}}
\newcommand{\Ukk}{\UU^{k+1}}
\newcommand{\Vk}{\VV^{k}}
\newcommand{\Vkk}{\VV^{k+1}}
\newcommand{\Tauk}{\Tau^{k+1}}
\newcommand{\vk}{\v^{k+1}}
\newcommand{\pk}{p^{k+1}}
\newcommand{\Dk}{\tensor{D\left( \vk \right)}}

\definecolor{lightgray}{gray}{0.9}
\definecolor{titlecolor}{RGB}{0,0,200}
\definecolor{subtitlecolor}{RGB}{0,0,153}
\definecolor{textcolor}{RGB}{0,128,255}
\definecolor{RoyalPurple}{RGB}{102,51,153}
\definecolor{ForestGreen}{RGB}{34,139,34}


\newcommand{\colbox}[1]{\colorbox{lightgray}{$ #1 $}}

\newcommand{\stitle}[2][0.3cm] { 
    {\normalsize \textcolor{titlecolor}{\textbf{#2}}} 
    \vspace{#1} 
}

\newcommand{\ssubtitle}[1]{ {\footnotesize \textcolor{subtitlecolor}{\textbf{#1}}} }

\newcommand{\stress}[1]{\textcolor{textcolor}{#1}}

\newcommand{\hidecontent}[2][0.25]{{% \hidecontent[<transparency>]{<stuff>}
  \setbox9=\hbox{#2}% Store <stuff> in \box9 to obtain height/width
  \transparent{#1}\ooalign{\usebox9\cr\color{white}\rule{\wd9}{\ht9}\cr}}}


\usetheme{Frankfurt}
\setbeamercovered{transparent}
\setbeamertemplate{bibliography item}[triangle]

\begin{document}

\title{\Large How to compute fast a function and all its derivatives}
\author[Keck]{\Large Jean-Baptiste Keck}
\institute[MSIAM]{\Large \bsc{M2 Msiam}}
\date{\large Paper Review\\ 02/02/2015}

\begin{frame}
    \titlepage
\end{frame}

\begin{frame}{Introduction}

    \begin{itemize} 
        \item In 1983, Walter Baur and Volker Strassen expose a lower bound for derivatives computations $\ccite{1}$.
        \item Proof is given but quite complex.
        \item No was algorithm exposed.
        \item Show that we can decuce interesting bounds.
    \end{itemize}

    \begin{block}{Theorem : Bauer-Strassen}
        The complexity of the evaluation of a rational function of several variables and all its derivatives is bounded above by three times the complexity of the evaluation of the function itself.
    \end{block}
\end{frame}

\begin{frame}{Contribution}
       
    Two years later in 1985, Jacques Moergenstern publish this paper \textit{How to compute fast a function and all its derivatives}.

    \vskip 0.5cm

    \begin{itemize} 
        \item Shows alternative proof which is much simpler
        \item Which is \alert{constructive} 
        \item Allows autodifferenciation techniques
    \end{itemize}
\end{frame}

\begin{frame}{Notations}

    \small
    \begin{itemize}
        \item $\mathbb{K}$ is an infinite field.\\
            \vskip 0.1cm
        \item$F \in \mathbb{K}_n(X)$ is a rational function of $n$ variables $x_1,x_2,...,x_n$.\\
        \item$\widetilde{F} \in \mathbb{K}_{n+1}(X)$ is a rational function of $n+1$ variables $x_1,x_2,...,x_n,y$.
            \vskip 0.3cm
        \item The set of partial derivatives of $F$ is denoted
            $\colbox{F' = \left \{ \pd{F}{x_1},\pd{F}{x_2}, ... ,\pd{F}{x_n} \right \}}$.
    \end{itemize}

    \vskip 0.3cm

    \begin{block}{Definition : Essential operation}
        Let \textbullseye\, be an arithmetical operation and $x = (x_1,x_2,...,x_n) \in \mathbb{K}^n$.\\
        \vskip 0.5cm
        $
        \left(\textrm{a \textbullseye\, b is an \textbf{essential operation}}\right)
        $
        \vskip 0.3cm
        $
        \Leftrightarrow
        \left\{
            \begin{array}{l}
                a = f(x)\textrm{ and }b = g(x) \textrm{ with } f,g \in \mathbb{K}_n(X) \\
                \textrm{\textbf{\alert{or} }}a \in \mathbb{K}, b = g(x) \textrm{ and \textbullseye\ is a division.} \\
            \end{array}
        \right.
        $
    \end{block}

\end{frame}



\begin{frame}{Algorithm}

\begin{itemize}
    \item Let $\A$ be an algorithm computing F from $x = (x_1,x_2,...,x_n)$ and $\mathbb{K}$.
\end{itemize}
    
\begin{block}{Theorem/Definition}
        $
        \exists\,\, u \in \mathbb{N} \,\,\, \A = \left\{ g_1,g_2,...,g_u \right\}
        \textrm{ with :} 
        $
        \hskip 0.3cm
        $
        \left\{
        \begin{array}{l}
        g_k \in \left\{x_1,x_2,...,x_n\right\} \cup \mathbb{K}\\
        \textrm{\textbf{\alert{or} }} g_k = g_{k_1} \textrm{ \textbullseye\, } g_{k_2}\textrm{ with }k_1,k_2 < k \textrm{, \textbullseye }\in \{ +,-,\times, \div \}
        \end{array}
        \right.
        $
\end{block}

\begin{block}{Example : $f(x) = f(x_1,x_2) = \frac{1+x_1^2}{x_2^4}, u = 8$}

{\centering% !
    $
    \left[
    \begin{array}{l c l}
        g_1 & = & 1 \\   
        g_2 & = & x_1 \\   
        g_3 & = & g_2 \times g_2 \\  
        g_4 & = & g_1 + g_3 \\
    \end{array}
    \right.
    \hskip 2cm
    \left[
    \begin{array}{l c l}
        g_5 & = & x_2 \\   
        g_6 & = & g_5 \times g_5 \\   
        g_7 & = & g_6 \times g_6 \\  
        g_8 & = & g_4 \div g_7 \\
    \end{array}
    \right.
    $
}

\end{block}
\end{frame}



\begin{frame}{Notations}

    \textbf{Finally let :}

\begin{itemize}
    \item \alert{$s(F)$} be the number of \textbf{essential multiplications and divisions} in $\A$.
        \vskip 0.5cm
    \item \alert{$m(F)$} be the total number of \textbf{multiplications and divisions} in $\A$.
        \vskip 0.5cm
    \item \alert{$T(F)$} the total number of \textbf{essential operations} in $\A$.
        \vskip 0.5cm
    \item \alert{$\Theta(F)$} the total number of \textbf{operations} in $\A$.
\end{itemize}

\end{frame}


\begin{frame}{Baur-Strassen's Theorem}

With all those notations :

\begin{block}{Baur-Strassen's Theorem}
From each algorithm $\A$ computing $F$ one can derive and algorithm $\Ar$ computing $F$ and $F'$ such that :

   $$
   (P) \Leftrightarrow
   \left\{
   \begin{array}{l c l}
       s(F,F') &\leq & \alert{\mathbf{3}} \,s(F)\\
       m(F,F') &\leq & \alert{\mathbf{3}} \,m(F)\\
       T(F,F') &\leq & \alert{\mathbf{5}} \,T(F)\\
       \Theta(F,F') &\leq & \alert{\mathbf{5}} \,\Theta(F)
   \end{array}
   \right.
   $$

   \vskip 0.3cm
   \textbf{Those inequalities are \alert{independent} of the number of variables $n$.}

\end{block}

\end{frame}


\begin{frame}{Proof overview}

    \begin{itemize}
        \item The proof is made by \textbf{induction on the length of the algorithm}.
    \end{itemize}

    \vskip 0.5cm
    Let \alert{$\Au$} be an algorithm of length u computing a rational function $F$ and $g_k$ be the result of the \alert{first operation} of $\Au$.
    \vskip 0.3cm
    Define \alert{$\widetilde{F}$} a function of $n+1$ variables such that $F(x_1,x_2,...,x_n)=\widetilde{F}(x_1,x_2,...,x_n,\mathbf{y})$ with $y = g_k(x_1,x_2,...,x_n)$.
    \vskip 0.3cm
    
    $\Au$ induces an algorithm \alert{$\A_{u-1}$} which computes $\widetilde{F}$ from $x_1,x_2,...,x_n,g_k$ and $\mathbb{K}$ in one less operation ($\A_{u-1}$ is of length $u-1$). \textbf{By induction hypothesis, $\widetilde{F}$ satisfies (P).}

\end{frame}

\begin{frame}{Example of induced algorithm}

    \begin{block}{Example : $F(x) = \frac{1+x_1^2}{x_2^4}\ \ \Rightarrow\ \ \widetilde{F}(x,g_3) = \frac{1+g_3}{x_2^4}$}

{\centering% !
    $
    \Au
    \hskip 0.5cm
    \Leftrightarrow
    \hskip 0.5cm
    \left [
    \begin{array}{l c l}
        g_1 & = & 1 \\   
        g_2 & = & x_1 \\   
        \alert{g_3} & \alert{=} & \alert{g_2 \times g_2} \\  
        g_4 & = & g_1 + g_3 \\
    \end{array}
    \right.
    \hskip 2cm
    \left[
    \begin{array}{l c l}
        g_5 & = & x_2 \\   
        g_6 & = & g_5 \times g_5 \\   
        g_7 & = & g_6 \times g_6 \\  
        g_8 & = & g_4 \div g_7 \\
    \end{array}
    \right.
    $
}

    $\Downarrow$
    
{\centering% !
    $ 
    \Auu
    \hskip 0.15cm
    \Leftrightarrow
    \hskip 0.5cm
    \left[
    \begin{array}{l c l}
        \g{1} & = & 1 \\   
        \textcolor{blue}{\g{2}} & \textcolor{blue}{=} & \textcolor{blue}{x_1} \\   
        \alert{\g{3}} & \alert{=} & \alert{g_3} \\
        \g{4} & = & \g{1} + \g{3} \\
    \end{array}
    \right.
    \hskip 2cm
    \left[
    \begin{array}{l c l}
        \g{5} & = & x_2 \\   
        \g{6} & = & \g{5} \times \g{5} \\   
        \g{7} & = & \g{6} \times \g{6} \\  
        \g{8} & = & \g{4} \div \g{7} \\
    \end{array}
    \right.
    $
}

\end{block}
\end{frame}

\begin{frame}{Proof overview}
    We have 
    $\colbox{\forall h \in \llbracket 1,n \rrbracket \,\,\, \pd{F}{x_h} = \pd{\widetilde{F}}{x_h} + \pd{\widetilde{F}}{y}\cdot\pd{g_k}{x_h}\ \ (1).
    }
    $

\vskip 0.5cm
\textbf{The idea is then to examine all six possible cases :}
\vskip 0.3cm

1. Case $\mathbf{g_k = \textcolor{red}{c \times x_i}}$ where $c \in \mathbb{K}$ and $i \in \llbracket 1,n \rrbracket$.
\vskip 0.1cm
2. Case $\mathbf{g_k = \textcolor{red}{x_i \times x_j}}$ where $i,j \in \llbracket 1,n \rrbracket$.
\vskip 0.1cm
3. Case $\mathbf{g_k = \alert{c \div x_i}}$ where $c \in \mathbb{K}$ and $i \in \llbracket 1,n \rrbracket$.
\vskip 0.1cm
4. Case $\mathbf{g_k = \alert{x_i \div x_j}}$ where $i,j \in \llbracket 1,n \rrbracket$.
\vskip 0.1cm
5. Case $\mathbf{g_k = \alert{x_i+c}}$ where $c \in \mathbb{K}$ and $i \in \llbracket 1,n \rrbracket$.
\vskip 0.1cm
6. Case $\mathbf{g_k = \alert{x_i+x_j}}$ where $i,j \in \llbracket 1,n \rrbracket$.

\end{frame}

\begin{frame}{Example case $g_k = c \times x_i$}
\end{frame}


\begin{frame}{References}
    %\raggedright
    %\nocite{*}
    %\bibliographystyle{plain}
    %\bibliography{master}
\end{frame}

\end{document}
