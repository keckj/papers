

\documentclass{beamer}

\usepackage[T1]{fontenc}
\usepackage[utf8]{inputenc}
\usepackage[french,english]{babel}

\usepackage{lmodern}
\usepackage{amsthm}
\usepackage{float}
\usepackage{lmodern}%pour un meilleur rendu des polices
\usepackage{verbatim}%du texte non interprt
\usepackage{amsmath}
\usepackage{amssymb}%maths
\usepackage{xspace}
\usepackage[dvipsnames,svgnames,table]{xcolor}
\usepackage{listings}
\usepackage{fancyhdr}
\usepackage{etoolbox}
\usepackage{titlesec}
\usepackage{titletoc}
\usepackage{lastpage}
\usepackage{hyperref}
\usepackage{ctable} % for \specialrule command
\usepackage{cite}
\usepackage{algorithm2e}
\usepackage{alltt}
\usepackage{array}
\usepackage{mdwmath}
\usepackage{mdwtab}
\usepackage{eqparbox}
\usepackage{subfig}
\usepackage{dblfloatfix}
\usepackage{url}
\usepackage{tipa}
\usepackage{stmaryrd}
\usepackage{upgreek}
\usepackage{mathrsfs}
\usepackage{ulem}
\usepackage{cancel}

\graphicspath{{img/}}
\DeclareGraphicsExtensions{.pdf,.jpeg,.jpg,.png}


\newcommand{\ccite}[1]{\textbf{\cite{#1}}}
\newcommand{\pd}[2]{\dfrac{\partial #1}{\partial #2}}
\newcommand{\od}[2]{\dfrac{\mathscr{D}_a #1}{\mathscr{D} #2}}
\newcommand{\tensor}[1]{\mathbf{#1}}
\renewcommand{\vector}[1]{\overrightarrow{#1}}

\newcommand{\Tau}{\tensor{\mathlarger{\uptau}}}
\renewcommand{\v}{\vector{u}}
\newcommand{\W}{\tensor{W\left( \v \right)}}
\newcommand{\D}{\tensor{D\left( \v \right)}}
\newcommand{\grad}{\vector{\nabla}}
\newcommand{\gradv}{\tensor{\nabla\v}}
\renewcommand{\div}[1]{div \left( #1 \right)}
\newcommand{\divv}[1]{\vector{div} \left( #1 \right)}
\newcommand{\UU}{\mathcal{U}}
\newcommand{\VV}{\mathcal{U}}
\newcommand{\M}{\mathcal{M}}
\newcommand{\A}[2]{\mathcal{A}_{#1}\left( #2 \right)}
\newcommand{\Uu}{\UU^{n}}
\newcommand{\Uv}{\UU^{n+\theta}}
\newcommand{\Uw}{\UU^{n+1-\theta}}
\newcommand{\Ux}{\UU^{n+1}}
\newcommand{\Uk}{\UU^{k}}
\newcommand{\Ukk}{\UU^{k+1}}
\newcommand{\Vk}{\VV^{k}}
\newcommand{\Vkk}{\VV^{k+1}}
\newcommand{\Tauk}{\Tau^{k+1}}
\newcommand{\vk}{\v^{k+1}}
\newcommand{\pk}{p^{k+1}}
\newcommand{\Dk}{\tensor{D\left( \vk \right)}}

\definecolor{lightgray}{gray}{0.9}
\definecolor{titlecolor}{RGB}{0,0,200}
\definecolor{subtitlecolor}{RGB}{0,0,153}
\definecolor{textcolor}{RGB}{0,128,255}
\definecolor{RoyalPurple}{RGB}{102,51,153}
\definecolor{ForestGreen}{RGB}{34,139,34}


\newcommand{\colbox}[1]{\colorbox{lightgray}{$ #1 $}}

\newcommand{\stitle}[2][0.3cm] { 
    {\normalsize \textcolor{titlecolor}{\textbf{#2}}} 
    \vspace{#1} 
}

\newcommand{\ssubtitle}[1]{ {\footnotesize \textcolor{subtitlecolor}{\textbf{#1}}} }

\newcommand{\stress}[1]{\textcolor{textcolor}{#1}}

\newcommand{\hidecontent}[2][0.25]{{% \hidecontent[<transparency>]{<stuff>}
  \setbox9=\hbox{#2}% Store <stuff> in \box9 to obtain height/width
  \transparent{#1}\ooalign{\usebox9\cr\color{white}\rule{\wd9}{\ht9}\cr}}}


\begin{document}
\pagestyle{fancy}

\maketitle

%\tableofcontents
%\clearpage

\section{Introduction}

This paper is about an alternative algorithmic proof of the Baur-Strassen result given and proved in \ccite{1},
showing that the complexity of the evaluation of a rational function of several variables and all its derivatives is bounded above by three times the complexity of the evaluation of the function itself. The new proof is constructive and much simpler than the original one in \ccite{1}.

\section{Notations}

The notation is mainly the one used in the original paper \cite{0} and \cite{1} with some minor changes.

\vskip 0.5cm
$\mathbb{K}$ is an infinite field.\\
\indent$F \in \mathbb{K}_n(X)$ is a rational function of $n$ variables $x_1,x_2,...,x_n$.\\
\indent$\widetilde{F} \in \mathbb{K}_{n+1}(X)$ is a rational function of $n+1$ variables $x_1,x_2,...,x_n,y$.
\vskip 0.3cm
\indent The set of partial derivatives of $F$ is denoted $F' = \left \{ \pd{F}{x_1},\pd{F}{x_2}, ... ,\pd{F}{x_n} \right \}.
\vskip 0.3cm
Let \textbullseye\, be an arithmetical operation and $x = (x_1,x_2,...,x_n) \in \mathbb{K}^n$.\\
$$
\left(\textrm{a \textbullseye\, b is an \textbf{essential operation}}\right)
\Leftrightarrow
\[
   \left \{
   \begin{array}{l}
       a = f(x)\textrm{ and }b = g(x) \textrm{ with } f,g \in \mathbb{K}_n(X) \\
       \textrm{\textbf{or }}a \in \mathbb{K}, b = g(x)\textrm{ and \textbullseye\, is a division.}
   \end{array}
   \right .
\]
$$

Let $\A$ be an algorithm computing F from $x = (x_1,x_2,...,x_n)$ and $\mathbb{K}$. We have :

$$
\begin{equation}
\colbox{
\label{algo}
\exists\,\, u \in \mathbb{N} \,\,\, \A = \left\{ g_1,g_2,...,g_u \right\}
\textrm{ with }
   \left \{
   \begin{array}{l}
       g_k \in \left\{x_1,x_2,...,x_n\right\} \cup \mathbb{K}\\
       \textrm{\textbf{or }} g_k = g_{k_1} \textrm{ \textbullseye\, } g_{k_2}\textrm{ with }k_1,k_2 < k
   \end{array}
   \right .
}
\end{equation}
$$

Finally let :

\begin{itemize}
\item $s(F)$ be the number of \textbf{essential multiplications and divisions} in $\A$.
    \item $m(F)$ be the total number of \textbf{multiplications and divisions} in $\A$.
    \item $T(F)$ the total number of \textbf{essential operations} in $\A$.
    \item $\Theta(F)$ the total number of \textbf{operations} in $\A$.
\end{itemize}

\section{Baur-Strassen's Theorem}
\textbf{Theorem:} From each algorithm $\A$ computing $F$ one can derive and algorithm $\Ar$ computing $F$ and $F'$ such that :


$$
\begin{equation}
    \label{SB}
    \colbox{
    (P) \Leftrightarrow
   \left \{
   \begin{array}{l c l}
       s(F,F') &\leq & \mathbf{3} \,s(F)\\
       m(F,F') &\leq & \mathbf{3} \,m(F)\\
       T(F,F') &\leq & \mathbf{5} \,T(F)\\
       \Theta(F,F') &\leq & \mathbf{5} \,\Theta(F)
   \end{array}
   \right .
}
\end{equation}
$$

\vskip 0.5cm
\textbf{Those inequalities are independent of the number of variables n.}

\section{Proof overview}

\textbf{Foreword:} \textit{Many errors have been corrected from the original paper and I tried to make some parts clearer so that it appears less confusing. However, new errors might as well have been introduced too.}

\vskip 0.2cm
The proof is made by \textbf{induction on the length of the algorithm}. Let $\Au$ be an algorithm of length u computing a rational function $F$ and $g_1$ be the result of the first operation of $\Au$.
\vskip 0.2cm
Define $\widetilde{F}$ a function of $n+1$ variables such that $F(x_1,x_2,...,x_n)=\widetilde{F}(x_1,x_2,...,x_n,\mathbf{y})$ with $y = g_1(x_1,x_2,...,x_n)$.
\vskip 0.2cm
$\Au}$ induces an algorithm $\A_{u-1}$ which computes $\widetilde{F}$ from $x_1,x_2,...,x_n,g_1$ and $\mathbb{K}$ in one less operation ($\A_{u-1}$ is of length $u-1$). \textbf{By induction hypothesis, $\widetilde{F}$ satisfies (P).}
\vskip 0.5cm
We have $\forall h \in \llbracket 1,n \rrbracket \,\,\, \pd{F}{x_h} = \pd{\widetilde{F}}{x_h} + \pd{\widetilde{F}}{y}\cdot\pd{g_1}{x_h}\ \ (1).$

\vskip 0.5cm
\textbf{The idea is then to examine all six possible cases :}
\vskip 0.3cm

1. Example case $\mathbf{g_1 = \textcolor{red}{c \cdot x_i}}$ where $c \in \mathbb{K}$ and $i \in \llbracket 1,n \rrbracket$ : 
$$
    (1) \Rightarrow 
    \[
   \left \{
   \begin{array}{l r}
   \pd{F}{x_h}(x_1,...,x_n) = \textcolor{purple}{\pd{\widetilde{F}}{x_h}(x_1,...,x_n) \,\mathbf{+}}\, \textcolor{blue}{c \cdot \pd{\widetilde{F}}{y}(x_1,...,x_n)} & \textrm{ if }h = i\\ \\
       \pd{F}{x_h}(x_1,...,x_n) = \pd{\widetilde{F}}{x_h}(x_1,...,x_n) & \textrm{ if } h \neq i
   \end{array}
   \right .
\]
$$

\vskip 0.5cm
$
\textrm{In this case we have :}
\[
   \left \{
   \begin{array}{l c l l l l l l l}
   s(F,F') & = & s(\widetilde{F}},\widetilde{F}') &&\\
   m(F,F') &\leq & m(\widetilde{F}},\widetilde{F}')&+&\textcolor{blue}{1}&+&\textcolor{red}{1}\\
   T(F,F') &\leq & T(\widetilde{F}},\widetilde{F}')&+&\textcolor{purple}{1}\\
       \Theta(F,F') &\leq & \Theta(\widetilde{F},\widetilde{F}')&+&\textcolor{purple}{1}&+&\textcolor{blue}{1}&+&\textcolor{red}{1}
   \end{array}
   \right .
\]
$

\vskip 0.5cm
2. Case $\mathbf{g_1 = \textcolor{red}{x_i \cdot x_j}}$ where $i,j \in \llbracket 1,n \rrbracket$.


$$
    (1) \Rightarrow 
    \[
   \left \{
   \begin{array}{l r}
   \pd{F}{x_h}(x_1,...,x_n) = \textcolor{purple}{\pd{\widetilde{F}}{x_h}(x_1,...,x_n) \,\mathbf{+}}\, \textcolor{blue}{x_j \cdot \pd{\widetilde{F}}{y}(x_1,...,x_n)} & \textrm{ if }h = i\\ \\
   \pd{F}{x_h}(x_1,...,x_n) = \textcolor{purple}{\pd{\widetilde{F}}{x_h}(x_1,...,x_n) \,\mathbf{+}}\, \textcolor{blue}{x_i \cdot \pd{\widetilde{F}}{y}(x_1,...,x_n)} & \textrm{ if }h = j\\ \\
       \pd{F}{x_h}(x_1,...,x_n) = \pd{\widetilde{F}}{x_h}(x_1,...,x_n) & \textrm{ if } h \notin \{i,j\}
   \end{array}
   \right .
\]
$$

\vskip 0.5cm
$
\textrm{In this case we have :}
\[
   \left \{
   \begin{array}{l c l l l l l l l}
   s(F,F') &\leq & s(\widetilde{F}},\widetilde{F}')&+&\textcolor{blue}{2}&+&\textcolor{red}{1}\\
   m(F,F') &\leq & m(\widetilde{F}},\widetilde{F}')&+&\textcolor{blue}{2}&+&\textcolor{red}{1}\\
       T(F,F') &\leq & T(\widetilde{F},\widetilde{F}')&+&\textcolor{purple}{2}&+&\textcolor{blue}{2}&+&\textcolor{red}{1}&
       \Theta(F,F') &\leq & \Theta(\widetilde{F},\widetilde{F}')&+&\textcolor{purple}{2}&+&\textcolor{blue}{2}&+&\textcolor{red}{1}
   \end{array}
   \right .
\]
$

\clearpage
\textbf{We do exactly the same for the 4 other cases :}
\vskip 0.5cm
    3. Case $\mathbf{g_1 = c/x_i}$ where c \in \mathbb{K}$ and $i \in \llbracket 1,n \rrbracket$.
\vskip 0.1cm
    4. Case $\mathbf{g_1 = x_i/x_j}$ where $i,j \in \llbracket 1,n \rrbracket$.
\vskip 0.1cm
    5. Case $\mathbf{g_1 = x_i+d}$ where $d \in \mathbb{K}$ and $i \in \llbracket 1,n \rrbracket$.
\vskip 0.1cm
    6. Case $\mathbf{g_1 = x_i+x_j}$ where $i,j \in \llbracket 1,n \rrbracket$.

\vskip 0.5cm
Putting everything together and using (P) for $\widetilde{F}$, we get at most the inequalities:
$$
\[
   \left \{
   \begin{array}{l c l l l c l l l c l}
   s(F,F') &\leq & s(\widetilde{F}},\widetilde{F}')&+&3&\leq&3\,s(\widetilde{F})&+&3&=&3\,s(F)\\
   m(F,F') &\leq & m(\widetilde{F}},\widetilde{F}')&+&3&\leq&3\,m(\widetilde{F})&+&3&=&3\,m(F)\\
   T(F,F') &\leq & T(\widetilde{F}},\widetilde{F}')&+&5&\leq&5\,T(\widetilde{F})&+&5&=&5\,T(F)\\
       \Theta(F,F') &\leq & \Theta(\widetilde{F},\widetilde{F}')&+&5&\leq&5\,\Theta(\widetilde{F})&+&5&=&5\,\Theta(F)
   \end{array}
   \right .
\] \Rightarrow \textrm{(P) for F, which proves the theorem since the beginning of the induction is trivial.}
$$

\vskip 0.5cm

Those inequalities are independent of the number of variables n since we just added a bounded number of new operations.

\section{Conclusion}

From this new proof of the theorem we can deduce a constructive recursive backward way to build an algorithm for the computation of all partial derivatives of $F$ at a given point $x$ from any algorithm computing $F$ at this point.

\vskip 0.3cm
This paper might be the starting point for autodifferentation techniques used to numerically evaluate the derivative of a function specified by a computer program.
Autodifferentiation exploits the fact that every computer program,
no matter how complicated,
executes a sequence of elementary arithmetic operations 
(addition,
subtraction,
multiplication,
division)
like in this paper but with additional elementary functions 
(exp,
log,
sin,
cos,
etc.).
By applying exactly the same chain rule repeatedly to these operations,
derivatives of arbitrary order can be computed automatically,
using at most a small constant factor more arithmetic operations than the original program.
Tools like Tapenade (developped at Inria) can return the forward tangent or reverse adjoint differentiated programs. 

\bibliographystyle{plain}
\bibliography{master}

\end{document}




 
 
 
 
 
 
 
 
 
 
 
 
 
